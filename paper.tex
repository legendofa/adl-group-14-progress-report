%%%%%%%%%%%%%%% PROGRESS REPORT TEMPLATE (ICML STYLE) %%%%%%%%%%%%%%%%%

\documentclass{article}

% Recommended packages
\usepackage{microtype}
\usepackage{graphicx}
\usepackage{subfigure}
\usepackage{booktabs}
\usepackage{amsmath,amssymb,mathtools,amsthm}
\usepackage[capitalize,noabbrev]{cleveref}
\usepackage{hyperref}

% ICML style (uncomment accepted if you want author names shown)
\usepackage{icml2025}
% \usepackage[accepted]{icml2025}

% Theorems (optional)
\theoremstyle{plain}
\newtheorem{theorem}{Theorem}[section]
\newtheorem{lemma}[theorem]{Lemma}
\theoremstyle{definition}
\newtheorem{definition}[theorem]{Definition}
\theoremstyle{remark}
\newtheorem{remark}[theorem]{Remark}

% Title and Author Info
\icmltitlerunning{Progress Report on [Project Title]}

\begin{document}

\twocolumn[
\icmltitle{Progress Report on [Your Project Title]}

\begin{icmlauthorlist}
\icmlauthor{Your Name}{yyy}
\end{icmlauthorlist}

\icmlaffiliation{yyy}{Department of [Your Department], [Your University], [City], [Country]}
\icmlcorrespondingauthor{Your Name}{your.email@university.edu}

\vskip 0.3in
]

\printAffiliationsAndNotice{} % No equal contribution needed

\begin{abstract}
This document serves as a progress report for the project titled “[Your Project Title].” 
It summarizes the problem being addressed, related work in the area, methodology developed so far, 
current progress, and planned next steps.
\end{abstract}

\section{Problem Definition}
Clearly state the research problem you are addressing.  
Explain the motivation and importance of this problem in the broader context.  
Provide any mathematical formulations, hypotheses, or goals if applicable.

\section{Related Works}
Summarize key prior research relevant to your problem.  
Highlight the main methods, results, or ideas that influence your approach.  
Discuss the limitations of existing works that your project aims to overcome.

\section{Methodology}
Describe your proposed approach in detail.  
Explain your model, algorithm, or theoretical framework.  
If applicable, include equations, diagrams, or pseudocode that clarify your method.

\section{Current Progress}
Summarize what has been accomplished so far.  
Include any preliminary results, experiments, or theoretical findings.  
Use tables or figures as needed.

\begin{table}[ht]
    \caption{Preliminary experimental results.}
    \label{tab:results}
    \vskip 0.15in
    \begin{center}
    \begin{small}
    \begin{sc}
    \begin{tabular}{lcc}
    \toprule
    Experiment & Metric & Result \\
    \midrule
    Exp. 1 & Accuracy & 0.87 \\
    Exp. 2 & RMSE & 1.23 \\
    \bottomrule
    \end{tabular}
    \end{sc}
    \end{small}
    \end{center}
    \vskip -0.1in
\end{table}

\section{Next Steps}
List your planned future tasks and timeline.  
Identify challenges, resources needed, or potential improvements.  
You may divide this section into short-term and long-term goals.

\subsection{Short-Term Goals}
\begin{itemize}
    \item Implement and test updated model parameters.
    \item Run additional experiments on larger datasets.
\end{itemize}

\subsection{Long-Term Goals}
\begin{itemize}
    \item Extend the method to real-world applications.
    \item Prepare the paper for submission to [Conference/Journal].
\end{itemize}

\section*{Acknowledgements}
(Optional) Acknowledge any collaborators, supervisors, or funding sources.

\bibliography{references}
\bibliographystyle{icml2025}

\end{document}
